\documentclass{article}
\usepackage[ngerman]{babel}
\usepackage[utf8]{inputenc}
\usepackage{fancyhdr}
\usepackage{libertine} 

\pagestyle{fancy}

\renewcommand{\headrulewidth}{0.4pt}
\renewcommand{\footrulewidth}{0.4pt}
\cfoot{Seite \thepage}

\title{Das Django Web-Framework dargestellt anhand des Praktischen Beispieles eines Inventarisierungssystemes}
\author{Clemens Dautermann}
\date{2. Januar 2019 bis \today}

\begin{document}

\maketitle
\newpage
\tableofcontents
\newpage


\section{Einleitung}
Diese Facharbeit soll einen grundlegenden Überblick über die wichtigsten Funktionen des Django Web-Frameworks geben.\newline
Das Django Web-Framework ist ein größtenteils in Python geschriebenes\footnote{Offizielle Django GitHub Seite https://github.com/django/django} Framework
zum entwickeln von Webservern. Es ist aufgrund seiner ausgeprägten Modularität und der Existenz einer Vielzahl von Datenbanktreibern besonders gut für
die Entwicklung von Webservern geeignet, die eine Datenbank erfordern.\newline 
Django stellt eine grundlegende Struktur für die Entwicklung zur Verfügung. So zum Beispiel:
\begin{itemize}
	\item Eine settings.py Die genutzt werden kann um Konfigurationsmöglichkeiten zentral zu bündeln
	\item Eine library um einfache Zugriffe auf Datenbanken zu tätigen
	\item Ein Routingsystem um eine einfachere Verwaltung von Urls zu gewährleisten
	
\end{itemize}

\end{document}
